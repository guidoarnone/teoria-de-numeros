\documentclass[11pt,a4paper]{amsart}
\usepackage[margin=1in]{geometry}

\usepackage{etoolbox}
\apptocmd{\sloppy}{\hbadness 10000\relax}{}{}
\newcommand{\edit}[1]{\marginpar{\footnotesize{#1}}}
\newcommand{\comment}[1]{}

%Text formatting and language configuration
\usepackage{microtype}
\usepackage[spanish]{babel}
\usepackage[utf8]{inputenc}
\usepackage[frenchstyle,largesmallcaps,light,oldstylemath,fulloldstylenums]{kpfonts}
\usepackage{enumitem}
\usepackage{amsmath,amsthm,amssymb,amsfonts}
\usepackage{tikz-cd}

%Semantic Macros
\newcommand{\C}{\mathbb{C}}
\newcommand{\R}{\mathbb{R}}
\newcommand{\Q}{\mathbb{Q}}
\newcommand{\Z}{\mathbb{Z}}
\newcommand{\N}{\mathbb{N}}
\newcommand{\A}{\mathbb{A}}
\newcommand{\F}{\mathbb{F}}
\renewcommand{\P}{\mathfrak{P}}
\newcommand{\nodiv}{{\not | \ }}
\renewcommand{\emph}[1]{\textbf{\textit{#1}}}
\newcommand{\xto}{\xrightarrow}

%Math Operators
\DeclareMathOperator{\coker}{coker}
\DeclareMathOperator{\colim}{colim}
\DeclareMathOperator{\disc}{disc}
\DeclareMathOperator{\rk}{rk}
\DeclareMathOperator{\End}{End}
\DeclareMathOperator{\Gal}{Gal}
\DeclareMathOperator{\Cl}{Cl}
\DeclareMathOperator{\car}{car}
\DeclareMathOperator{\adj}{adj}
\DeclareMathOperator{\mcd}{mcd}
\DeclareMathOperator{\Pic}{Pic}
\DeclareMathOperator{\ICM}{ICM}
\DeclareMathOperator{\M}{M}
\DeclareMathOperator{\GL}{GL}
\DeclareMathOperator{\SL}{SL}
\DeclareMathOperator{\CL}{Cl}

\usepackage{listings}
\usepackage{xcolor}

\definecolor{thmcol}{RGB}{135, 24, 107}
\definecolor{citecol}{RGB}{135, 24, 107}
\definecolor{linkcol}{RGB}{135, 24, 107}
\definecolor{urlcol}{RGB}{135, 24, 107}

\numberwithin{equation}{section}

\theoremstyle{plain}
\newtheorem*{que*}{Pregunta}
\newtheorem{introthm}{Teorema}
\newtheorem{ej}[equation]{Ejemplo}
\newtheorem{thm}[equation]{Teorema}
\newtheorem{prop}[equation]{Proposición}
\newtheorem{lem}[equation]{Lema}
\newtheorem{obs}[equation]{Observación}
\newtheorem{coro}[equation]{Corolario}

\renewcommand\qedsymbol{\Large\color{thmcol}${\divideontimes}$}

\usepackage[colorlinks=true, pagebackref=true]{hyperref}
\hypersetup{
    colorlinks,
    citecolor=citecol,
    linkcolor=linkcol,
    urlcolor=urlcol
}
\usepackage{amsrefs}
\usepackage{url}


\renewcommand{\SS}{\mathscr{S}}
\renewcommand{\O}{\mathcal{O}}
\newcommand{\fQ}{\mathfrak{Q}}
\newcommand{\fR}{\mathfrak{R}}

\usepackage{tcolorbox}

%\usepackage[notcite,notref]{showkeys}

\title[El problema de conjugación para matrices enteras]{El problema
de conjugación para matrices enteras}
\author{G. A.}

\date{}

\begin{document}

\maketitle

El puntapié inicial de las presentes notas es la siguiente pregunta:


\begin{tcolorbox}[colback=thmcol!30, sharp corners, colframe=white]
\begin{center}
¿Dadas dos matrices $A, B \in M_n \Z$, existe $P \in \GL_n \Z$
tal que $B = PAP^{-1}$?
\end{center}
\end{tcolorbox}

Si la respuesta a la pregunta para determinadas matrices
$A$ y $B$ es afirmativa, decimos que son \emph{conjugadas}.
Una forma de interpretar la pregunta es la siguiente: el grupo general
lineal $\GL_n \Z$ actúa en $\M_n \Z$ por conjugación, esto es,
via $P\cdot A = PAP^{-1}$. Estamos interesados en entender las órbitas
de esta acción, i.e. las clases de conjunto cociente $M_n \Z/GL_n \Z$.

Un invariante de la clase de conjugación
de una matriz es su polinomio característico: esto se deduce de que
\begin{align*}
\chi_{PAP^{-1}} &= \det(X \cdot I - PAP^{-1}) =
\det(X \cdot PIP^{-1} - PAP^{-1}) \\&= \det(P)\det(X \cdot I - A)\det(P)^{-1}
\\&= \det(X \cdot I - A) = \chi_A.
\end{align*}
En el caso en el que el polinomio caracterítstico es irreducible,
Latimer y MacDuffee parametrizan las clases de conjugación
de matrices en términos de ideales
fraccionarios:

\begin{tcolorbox}[colback=thmcol!30, sharp corners, colframe=white]
\begin{introthm}[\cite{lm}*{Theorem}]\label{thm:main} Sea $\alpha$
un entero algebraico y $\mathcal O = \Z[\alpha]$. Existe una correspondencia
biyectiva entre ideales fraccionarios de $\mathcal O$ y clases de
conjugación de matrices enteras con polinomio característico $m(\alpha, \Q)$.
\end{introthm}
\end{tcolorbox}

Para probar el teorema introducimos la noción de orden
e ideal fraccionario de un orden en la Sección \ref{sec:ord}.
En la Sección \ref{sec:main} probarmos el Teorema \ref{thm:main}
y damos algunos ejemplos breves del algoritmo que ofrece la demostración.
Por último, en la Sección \ref{sec:SL} tratamos el problema de conjugación
por matrices de determinante $1$ y lo relacionamos con los elementos de norma
positiva, los totalmente positivos y el \textit{narrow-class group}.

\section{Órdenes e ideales fraccionarios}\label{sec:ord}

Sea $K$ una extensión finita de $\Q$ y $\O_K$ su anillo de enteros. Un
\emph{orden} de $K$ es un subanillo que como $\Z$-módulo tiene
rango $[K:\Q]$. Notar que $\mathsf{Frac}(\mathcal O) = K$, ya que el primer
cuerpo está contenido en el segundo y ambos tienen la misma
$\Q$-dimensión. Un \emph{$\mathcal O$-ideal fraccionario} es
un $\mathcal O$-módulo $I \subset K$; siempre existe $x \in \mathcal O$
y un ideal $J \trianglelefteq \mathcal O$ tal que $I = \frac{1}{x} J$.
\begin{ej} Si $\alpha$ es un entero algebraico y $K=\Q(\alpha)$,
entonces $\Z[\alpha]$ es un orden de $\O_K$. En particular
$\Z[\sqrt{-5}]$ es un orden de $\Q(\sqrt{-5})$ que está contenido
propiamente en su anillo de enteros $\Z[\frac{1+\sqrt{-5}}{2}]$.
\end{ej}
Dos $\mathcal O$-ideales fraccionarios $I$ y $J$ se dicen \emph{equivalentes}
si $I = xJ$ para algún $x \in K \setminus \{0\}$. Esta es una relación
de equivalencia; notamos $\ICM(\mathcal O)$ al conjunto de clases de
equivalencia de ideales fraccionarios. La multiplicación de ideales define
una estructura de monoide en este conjunto; notamos $\Pic(\O)$
al grupo de elementos inversibles de $\ICM(\O)$. En general,
si $\mathcal O \ne \mathcal O_K$, no todo
ideal fraccionario es inversible.

Precisaremos los siguientes lemas sobre ideales fraccionarios
más adelante.

\begin{lem} \label{lem:hom=units}
Sea $K$ una extensión finita de $\Q$ y
$\mathcal O$ un orden de $K$.
Dos $\O$-ideales fraccionarios $I$
y $J$ son equivalentes si y sólo si son isomorfos como $\O$-módulos.
Más aún, isomorfismo $\O$-lineal $I \to J$ está dado por la multiplicación
por un elemento de $K\setminus \{0\}$.
\end{lem}
\begin{proof} Si $I = xJ$ para cierto $x \in K \setminus \{0\}$,
el morfismo $j \in J \mapsto xj \in I$ resulta un isomorfismo $\O$-lineal.
Recíprocamente, supongamos que tenemos un isomorfismo $\O$-lineal
$\varphi \colon I \to J$. Por la implicación ya demostrada, podemos suponer
que $I, J \subset \O$, es decir que $I$ y $J$ son ideales de $\O$.
Ahora, dado $x \in I$ no nulo, para cada $i \in I$ es
\[
\varphi(x)i = \varphi(xi) = x\varphi(i).
\]
Esto implica que $\varphi$ coincide con el morfismo dado por la multiplicación
por $\varphi(x)/x$. En particular tomando imágenes es
$J = \frac{\varphi(x)}{x} I$.
\end{proof}

\begin{lem} \label{lem:frac-range}
Sea $K$ una extensión finita de $\Q$. Si $\O$ es
un orden de $K$, todo $\O$-ideal fraccionario no nulo
tiene rango $[K:\Q]$ como $\Z$-módulo.
\end{lem}
\begin{proof} Sea $I$ un $\O$-ideal fraccionario, que salvo
isomorfismo $\O$-lineal (en particular, $\Z$-lineal) podemos suponer
contenido en $\O$. Tensorizando por $\Q$ a la sucesión exacta
$0\to I \hookrightarrow \O \twoheadrightarrow \O/I \to 0$
vemos que $\rk I = \rk \O$ si y sólo si $\rk \O/I = 0$.
Para ver esto último probaremos que $\O/I$ es finito:
dado $x \in I$ no nulo tenemos un epimorfismo
$\O/x \O \to \O/I$; podemos asumir entonces que $I = (x)$.
Finalmente, el mismo argumento que en el caso $\O = \O_K$ prueba que
el cociente $\O/x \O$ tiene cardinal $N_{K/\Q}(x)$.
\end{proof}

\section{El teorema de Latimer-MacDuffee}\label{sec:main}

En esta sección probamos el Teorema \ref{thm:main}.
De aquí en más fijamos $\alpha$ un entero algebraico con polinomio minimal $f$
de grado $n$ y notemos $\mathcal O = \Z[\alpha]$ y $K = \Q(\alpha)$.

Observemos que para todo $\O$-ideal fraccionario $I$
la multiplicación por $\alpha$ define un morfismo
$\Z$-lineal,
\[
m_I \colon I \to I, \qquad x \mapsto \alpha x.
\]
Por el Lema \ref{lem:frac-range}, todo tal ideal $I$ es $\Z$-libre
de rango $n$; en paticular, dada una $\Z$-base $B$ de $I$, podemos considerar
la matriz $[L_I]_B \in \M_n \Z$ de $L_I$ en base $B$.
Si cambiamos la base por otra, digamos $B'$, entonces $[L_I]_B$
y $[L_I]_{B'}$ son conjugadas con matriz de conjugación
la matriz de cambio de base $C_{B,B'}$.

Por otro lado, si $J$ es un $\O$-ideal fraccionario equivalente a $I$,
por el Lema \ref{lem:hom=units} esto equivale a tener un
isomorfismo $\O$-lineal $\phi \colon I \to J$
dado por la multiplicación
por cierto elemento $\beta \in K \setminus \{0\}$. Se sigue de aquí que
$\varphi m_I = m_J \varphi$, pues
\[
\varphi(m_I(x)) = \varphi(\alpha x) = \beta \alpha x
= \alpha \beta x = m_J(\varphi(x))
\]
para todo $x \in I$. En particular, dadas $\Z$-bases $B$ de $I$ y $B'$ de $J$,
las matrices $[L_I]_{B}$ y $[L_J]_{B'}$ serán conjugadas con matriz de
conjugación $[\varphi]_{B, B'}$. (También se puede observar que si $J = xI$
para cierto $x \in K \setminus \{0\}$, entonces $[L_J]_{xB} = [L_I]_B$.)

Si $I$ es un $\O$-ideal fraccionario,
vamos a notar $[L_I]$ a la clase de conjugación de las matrices $[L_I]_B$
donde $B$ es una $\Z$-base de $I$. El conjunto de matrices de $\M_n \Z$
de polinomio característico $f$ será denotado $\M_f$; recordemos que
$GL_n(\Z)$ actúa allí por conjugación.
La discusión anterior prueba la siguiente proposición.

\begin{prop} Se tiene una función bien definida
\begin{equation}\label{def:lambda}
\Lambda \colon \ICM(\O) \to \M_f/\GL_n(\Z), \qquad [I] \mapsto [L_I].
\end{equation}
\qed
\end{prop}

El Teorema \ref{thm:main} será una consecuencia de que la función $\Lambda$
es biyectiva, como veremos a continuación.

\begin{prop} La función \eqref{def:lambda} es sobreyectiva.
\end{prop}
\begin{proof} Sea $A \in M_f$ y veamos que existe un $\O$-ideal
fraccionario tal que $\Lambda([I]) = A$. Como $\O = \Z[X]/(f)$,
y $f(A) = 0$ por el teorema de Cayley-Hamilton,
la multiplicación por $A$ define una estructura de $\O$-módulo
en $N:= \Z^n$ donde la multiplicación por $\alpha$ se identifica con
la multiplicación por $A$. Más aún, esta estructura
es una restricción de la estructura
de $K$-módulo que $A$ define sobre $M := \Q^n$.

Observemos que
\[
n = \dim_{\Q} M = \dim_K M \cdot \dim_\Q K = \dim_K M \cdot n,
\]
así que $\dim_K M = 1$. En consecuencia, existe un isomorfismo
$K$-lineal $\varphi \colon M \to K$, que se restringe
entonces a un isomorfismo $\O$-lineal $\varphi| \colon N \to \varphi(N)$.
Por definición $I := \varphi(N)$ es un $\O$-ideal fraccionario y
la multiplicación por $\alpha$ en $I$ tiene matriz $A$ en base
$\{\varphi(e_1), \ldots, \varphi(e_n)\}$. En particular
$\Lambda([I]) = [A]$.
\end{proof}

\begin{prop} La función \eqref{def:lambda} es inyectiva. \label{prop:lambda-mono}
\end{prop}
\begin{proof} Supogamos que $[L_I] = [L_J]$, de forma que existen
una matriz $U \in \GL_n \Z$ y $\Z$-bases $B$ de $I$ y $B'$ de $J$
tal que $U [L_I]_{B} = [L_J]_{B'} U$. La matriz $U$ define
un isomorfismo $\Z$-lineal $I \to J$ que, al conmutar con la multiplicación
por $\alpha$, es además $\O$-lineal. Por el Lema \ref{lem:hom=units},
se tiene entonces que $[I] = [J]$.
\end{proof}

\subsection{De matrices a ideales} Si $\O_K$ es monogenerado
y $\CL(\O_K) = 1$, todo par de matrices con polinomio característico
$f$ son conjugadas. Hagamos un ejemplo no trivial.

Consideremos $d = -5$ y $K = \Q(\sqrt{-5})$, $\O = \O_K = \Z[\sqrt{-5}]$.
Como consecuencia de la cota de Minkowski,
el grupo de clases de $\O$ es isomorfo a $\Z_2$, generado por
$I = (2,1+\sqrt{5})$. Tomemos como $\Z$-base de $(1)$ a
$\{1,\sqrt{-5}\}$. De esta forma, la multiplicación de $\sqrt{-5}$
tiene en esta base matriz $A_0 =
\begin{pmatrix}0 & -5\\ 1 & 0\end{pmatrix}$.
Para $I$ tomamos la $\Z$-base $\{2,1+\sqrt{5}\}$;
como $2\sqrt{-5} = -2 + 2(1+\sqrt{-5})$ y $\sqrt{-5}(1+\sqrt{-5}) = -5 +
\sqrt{-5} = -3 \cdot 2 + (1+\sqrt{-5})$, la multiplicación
por $\sqrt{-5}$ en esta base tiene matriz $A_1 = \begin{pmatrix}
-1 & -3\\ 2 & 1 \end{pmatrix}$.

Por el Teorema \ref{thm:main}, toda matriz entera $A$
de $2 \times 2$ que satisfaga $A^2 = -5 I$ es conjugada a $A_0$ ó $A_1$,
y estas dos últimas no son conjugadas.

\subsection{De ideales a matrices}

Vamos ahora en la dirección opuesta, consideremos la matriz
$A = \begin{pmatrix}2 & 3 \\ -3 & -2 \end{pmatrix}$ que
satisface la ecuación $A^2 + 5 I = 0$. La multiplicación
por $A$ define una estructura de $\Q(\sqrt{-5})$-módulo
en $\Q^2$ y de $\Z[\sqrt{-5}]$-módulo en $\Z^2$. Concretamente
\[
(a+b\sqrt{-5})\begin{pmatrix}x\\ y\end{pmatrix}
= (aI+bA)\begin{pmatrix}x\\ y\end{pmatrix}
= \begin{pmatrix}(a+2b)x+3by\\-3bx + (a-2b)y\end{pmatrix}.
\]
Tomando por ejemplo $v_0 =\begin{pmatrix} 1 \\ 0\end{pmatrix}$,
tenemos un isomorfismo mandando $\gamma \in K \mapsto \gamma v_0$, i.e.
\[
\varphi \colon
a+b\sqrt{-5} \in K \longmapsto \begin{pmatrix}a+2b\\-3b\end{pmatrix} \in \Q^2.
\]
Luego la preimagen de $\Z^2$ son los elementos $a+b\sqrt{-5}$
tales que $a+2b, 3b \in \Z$. Si ponemos $b = -k/3$ para cierto $k \in \Z$,
y $a = l+2k/3$ para cierto $l \in \Z$, entonces $a+b\sqrt{-5}
= l + k(2/3-\sqrt{-5}/3)$. Luego
\[
I := \varphi^{-1}(\Z^2) =
\Z + \left(\frac{2}{3}-\frac{1}{3}\sqrt{-5}\right)
\]
Multiplicando por $3$ obtenemos la misma clase, así que la matriz $A$
está asociado a la clase del ideal fraccionario $3\Z+(2-\sqrt{-5})\Z
= (3,2-\sqrt{-5})$.

Para ver si $A$ es conjugada a $A_0$ o a $A_1$,
basta analizar si $(3,2-\sqrt{-5})$ es principal. El cociente
$\Z[\sqrt{-5}]/(3,2-\sqrt{-5})$ es isomorfo a $\Z/3\Z$; de ser
princial este ideal existiría $x \in \Z[\sqrt{-5}]$ de norma $3$;
i.e. enteros $a$ y $b$ tales que $a^2+5b^2 = 3$. Sin embargo,
para que suceda esto debe ser $b = 0$ y luego $3 = a^2$, lo cual
es absurdo pues $3$ no es un cuadrado.

En definitiva, se obtuvo que $(3,2-\sqrt{-5}) \sim (2,1+\sqrt{5})$
y entonces $A$ es conjugada a $A_1$. Más todavía, podemos
explicitar la matriz de conjugación. En pos de la brevedad,
referimos a \cite{KCd}*{Example 3.5} para el resto de los cálculos.

\section{Conjugación por matrices de determinante 1}\label{sec:SL}

Refinamos ahora nuestra pregunta inicial. Si $UA = BU$
para ciertas matrices $A,B \in \M_n \Z$ y $U \in \GL_n \Z$,
entonces $\det U = \pm 1$.

\begin{tcolorbox}[colback=thmcol!30, sharp corners, colframe=white]
\begin{center}
¿Cuándo son dos matrices enteras
conjugadas por una matriz de determinante $1$?
\end{center}
\end{tcolorbox}

La relación de conjugación por $\SL_n \Z$
es igual o más fina que la conjugación por $\GL_n \Z$.
El siguiente lema muestra que para cada clase de conjugación
$\GL_n \Z$ hay a lo sumo dos clases de conjugación por $\SL_n \Z$.

\begin{lem} \label{lem:sl=gl}
Sea $A \in \M_n \Z$ y $D \in \GL_n \Z$ tal que $\det D = -1$ (
por ejemplo $D=\mathrm{diag}(-1,1,\ldots,1)$).
\begin{itemize}
  \item[(i)] Si $B$ es conjugada a $A$ por una matriz de $\GL_n \Z$,
  está en la clase de conjugación por $\SL_n \Z$ de $A$ ó $DAD^{-1}$.
  \item[(ii)] Existe $U \in \SL_n \Z$ tal que $UAU^{-1} = DAD^{-1}$
  si y sólo si existe una matriz $C \in \GL_n \Z$ que conmuta con $A$ y tiene
  determinante $-1$.
\end{itemize}
\end{lem}
\begin{proof} Si $A = UBU^{-1}$ para cierta $U$ inversible, entonces
o bien $\det U = 1$ y entonces $B$ está en la clase de $\SL_n \Z$-conjugación
de $A$ o bien $\det U = -1$ y luego $\det(DU) = 1$,
$(DU)B(DU)^{-1} = DAD^{-1}$. Esto prueba (i).

Para ver (ii), obserevemos que $DAD^{-1} = U^{-1}AU$ para cierta matriz
de determinante $1$ si y sólo si $(U^{-1}D)A = A(U^{-1}D)$.
La multiplicación a derecha por $D$ establece una biyección entre
$\SL_n \Z$ y $-\SL_n \Z$, por tanto, la ecuación de arriba se satisface
sólo si existe $C \in -\SL_n \Z$ tal que $CA = AC$.
\end{proof}

\begin{obs} Como $I_n \in \Z(\M_n \Z)$
y $\det I_n = -1$ si $2 \not \mid n$, para $n$ impar el problema
de conjugación para $\SL_n \Z$ coincide con el problema para $\GL_n \Z$.
\end{obs}

Podemos caracterizar precisamente cuándo las clases de conjugación
de $\SL_n \Z$ y $\GL_n \Z$ en $M_f$ coinciden.

\begin{prop} \label{prop:uni-sl=gl} Las siguientes afirmaciones son equivalentes:
  \begin{itemize}
    \item[i)] Existe $u \in \O^\times$ tal que $N_{K/\Q}(u) = -1$.
    \item[ii)] Para toda $A \in M_f$, se tiene que $\GL_n \Z \cdot A =
    \SL_n \Z \cdot A$.
    \item[iii)] Existe $A \in M_f$ tal que $\GL_n \Z \cdot A =
    \SL_n \Z \cdot A$.
\end{itemize}
\end{prop}
\begin{proof} Probemos que (i) implica (ii). Salvo conjugación,
el Teorema \ref{thm:main} nos permite tomar $A$ como la matriz
de multiplicación por $\alpha$ de un $\O$-ideal fraccionario $I$
en cierta $\Z$-base $B$. Si $D$ es la matriz de multiplicación por
$u$ como endomorfismo $\Z$-lineal de $I$ en base $B$, se tiene
que $\det(D) = N(u) = -1$ y $D$ conmuta con $A$; resta
aplicar el Lema \ref{lem:sl=gl?}.

Para probar que (ii) implica (iii)
no hay nada que decir; probemos que (iii) implica (i).
Como antes, si (iii) vale para una matriz $A$ arbitraria,
vale para cierta matriz de multiplicación por $\alpha$ para determinados
$\O$-ideal fraccionario $I$ en una $\Z$-base $B$. Por el
lema \ref{lem:sl=gl}, existe $D$ invertible
que conmuta con $[L_\alpha]_B$ y tiene determinante $-1$. La condición
de conmutar con la multiplicación
por $\alpha$ hace que la función $\varphi \colon I \to I$ asociada a $D$
sea $\O$-lineal; en consecuencia viene de multiplicar por cierto
$u \in K^times$. Como $uI = \varphi(I) = I$, es $u \in \O^\times$
y $N_{K/\Q}(u) = \det D = -1$.
\end{proof}

\subsection{El \emph{narrow-class group}}

Un elemento $x \in K$ se dice \emph{positivo}
si $N_{K/\Q}(x) > 0$ y \emph{totalmente positivo}
si $\sigma(x) > 0$
para todo embedding real $\sigma \colon K \to \R$.
Notamos $K^+ \subset K$
al conjunto de elementos totalmente positivos y
$\O^+ = \O \cap K^+$.
Se define el \emph{narrow-class group} de $\O$ como
el grupo $\Cl^+(\O)$ de $\O$-ideales fraccionarios inversibles
módulo la relación de
equivalencia $I \sim J \iff I = xJ$ para cierto $x \in K$ totalmente
positivo. Decimos en tal caso que $I$ y $J$ son
\emph{estrechamente equivalentes}.

\begin{obs} Observemos que $N_{K/\Q}(x)$ se puede ver como el
producto de las imagenes de $x$ a través de cada
\textit{embedding} $\sigma \colon K \to \C$. Algunos de ellos
se pueden correstringir a $\R$. Si $\sigma$ es un embedding complejo,
también lo es $\overline \sigma$, y entonces
$\sigma(x)\overline\sigma(x) = |\sigma(x)|^2 \ge 0$. El signo
de la norma depende entonces únicamente de los embeddings reales; en
particular, un elemento totalmente positivo es positivo.
\end{obs}

\begin{prop} \label{prop:sec} Se tiene una sucesión exacta corta
\[
1 \to \O^\times/\O^+ \to K^\times/K^+ \to
\CL^+(\O) \to \CL(\O) \to 1.
\]
\end{prop}
\begin{proof} Toda clase $[I]$ de ideal fraccionario
en $\CL(\O)$ es imagen de la clase de igual representante en $\CL^+(\O)$;
esto define un epimorfismo $\pi \colon \CL^+(\O)\to\CL(\O)$. El núcleo consiste
de las clases $[x \O]$ donde $x \in K^\times$. En particular se tiene
un morfismo $x\in K^\times/K^+ \mapsto [x \O] \in \CL^+(\O)$.
Su núcleo son las clases $[x] \in K^\times/K^+$ que satisfacen
$[x\O] = [\O]$, esto es, que existe $y \in K_+$ tal que $x \O = y \O$;
en particular $x/y \in \O$ y de forma simétrica $y/x \in \O$.
Por lo tanto $x = y \cdot z$ con $z \in \O^\times$
y la clase de $x$ en $K^\times/K^+$ pertenece a $\O^\times$. Finalmente
el núcleo del morfismo $\O^\times \to K^\times/K^+$ inducido por la inclusión
$\O^\times \subset K^\times$ es $\O^\times \cap K^+ = \O^+$.
\end{proof}

\begin{obs} Observemos que si $x \in K^\times$, entonces $x^2 \in K^+$.
En particular $K^\times/K^+$ y $\O^\times/O^+$ son $2$-grupos.
\end{obs}

\subsection{El grupo de clases módulo ideales principales de norma positiva}

Sea $NP(K) \subset K^\times$ el subconjunto de elementos de norma positiva,
y $NP(\O) = \O \cap NP(K)$. Consideramos $Cl_{NP}(\O)$ al grupo de ideales
fraccionarios invesibles módulo los generador por elementos de norma positiva.
Un argumento similar al de la Proposición
\ref{prop:sec} prueba que existe una sucesión exacta corta
\[
1 \to O^\times/NP(\O) \to K^\times /NP(K) \to \CL_{NP}(\O) \to \CL(\O) \to 1
\]
y $NP(K) = \ker(K^\times \xto{N} \Q^\times \xto{\mathrm{sgn}}
\{-1,1\})$, $NP(\O) = \ker(O^\times \xto{N} \Z^\times \xto{\mathrm{sgn}}
\{-1,1\})$. Se tiene por lo tanto el siguiente resultado:
\begin{prop}\label{prop:equiv-clnp=cl}
Se tiene que $|\CL_{NP}(\O)|/|\CL(\O)| \le 2$. Además, son equivalentes:
\begin{itemize}
  \item[i)] La sobreyección canónica $\CL_{NP}(\O) \to \CL(\O)$ es
  un isomorfismo.
  \item[ii)] Se tiene que $NP(K) = K^\times$ ó $NP(\O) \neq O^\times$.
\qed
\end{itemize}
\end{prop}

\subsection{Ejemplos en el caso cuadrático} De aquí en más fijamos
la siguiente notación: sea $d$ un entero positivo
libre de cuadrados y
$d \not \equiv 1 \pmod{4}$. Tomando $K = \Q(\sqrt{d})$, es $\O = \O_K =
\Z[\sqrt{d}]$. Los \textit{embedding} reales $\sigma_1, \sigma_2 \colon K
\to \R$ son
\[
\sigma_1(a+b\sqrt{d}) = a+b\sqrt{d}, \qquad
\sigma_2(a+b\sqrt{d}) = a-b\sqrt{d}.
\]
Por el teorema de unidades de Dirichlet, sabemos que $\mathcal O^\times
= \langle\pm 1\rangle \times \langle u \rangle$ con $u$ una unidad
fundamental. Podemos suponer, cambiando $u$ por $-u$, que $\sigma_1(u) > 0$;
de aquí se seguirá también que $\sigma_1(u)^n > 0$ para todo $n \in \Z$.
Observemos que en este caso
dos $\O$-ideales fraccionarios $I$ y $J$
son estrechamente equivalentes si y sólo
si $I = xJ$ para un elemento positivo $x$, ya que si $x$ es positivo
entonces $x$ ó $-x$ son totalmente positivos.

\begin{prop} El cociente $\O^\times/\O^+$ es isomorfo a $\Z/2\Z$
si $N(u) = 1$ e isomorfo a $(\Z/2\Z)^2$ si $N(u) = -1$.
\end{prop}
\begin{proof}
Notar que $N(u) = 1$ si y sólo si $u$ es totalmente positiva
- es decir, si $\sigma_2(u)>0$. Observemos que
\[
\sigma_i(\pm u^n) = \pm \sigma_i(u)^n.
\]
Si $i = 1$, esta expresión es positiva sólo si $\pm = 1$. Para que además
la expresión sea positiva si $i = 2$, debe ser o bien $\sigma_2(u) > 0$
o bien $n$ par. Si $N(u) = 1$, entonces $\sigma_2(u) > 0$ y $\O^+
= \langle 1 \rangle \times \langle u \rangle$, en cambio si $N(u) = -1$
entonces $\O^+ = \langle 1 \rangle \times \langle u^2 \rangle$.
\end{proof}

\begin{prop} El morfismo
\[
K^\times \xto{(\sigma_1,\sigma_2)}
\Q^\times \times \Q^\times \xto{\mathrm{sgn} \ \times \ \mathrm{sgn}} \{\pm 1\}^2
\]
es sobreyectivo y su núcleo es $K^+$.
En particular $K^\times/K^+ \simeq (\Z/2\Z)^2$.
\end{prop}
\begin{proof} Basta notar que las imagenes de $\sqrt{d}$ y $\sqrt{-d}$
son $(1,-1)$ y $(1,-1)$ respectivamente.
\end{proof}

\begin{obs} En general para todo cuerpo de números
el cociente $K^\times / K^+$ es isomorfo a un producto
de tantas copias de $\Z/2\Z$ como embeddings $K \to \R$,
ver \cite{ft}*{II.2.14}.
\end{obs}

Recordemos que se definen
$h_K = |\CL(\O_K)|$ y $h_K^+ = |\CL^+(\O_K)|$.

\begin{coro} Si $N(u) = -1$, entonces $\CL(\O) = \CL^+(\O)$. En caso
contrario el subgrupo de $\CL^+(\O)$ generado por los ideales
fraccionario principales tiene órden $2$, y
$|CL^+(\O)| = 2|\CL(\O)|$.
\qed
\end{coro}

\begin{center}
\begin{tabular}{c | c | c | c | c}
n &  unidad fundamental $u$ de $K = \Q(\sqrt{n})$ & $N_{K/\Q}(u)$ & $h_K$ & $h_K^+$\\
\hline
2 & 1 + $\sqrt{2}$ & -1 & 1 & 1 \\
\hline
3 & 2 + $\sqrt{3}$ & -1 & 1 & 1 \\
\hline
6 & 5 + 2$\sqrt{6}$ & 1 & 1 & 2 \\
\hline
7 & 8 + 3$\sqrt{7}$ & 1 & 1 & 2 \\
\hline
10 & 3 + $\sqrt{10}$ & -1 & 2 & 2 \\
\end{tabular}
\end{center}

\subsection{Un refinamiento del teorema de Latimer-MacDuffee}

Fijemos un conjunto de representantes $\{I_1,\ldots, I_k\}$ de
elementos de $\CL_{NP}(\O)$ por ideales de $\O$. Todo $\O$-ideal fraccionario
es de la forma $r I_j$ para cierto $r \in K$ $j \in \{1,\ldots,k\}$.

Para cada $j$, fijamos dos $\Z$-bases $B^+_j, B^+_j$ tales que
la matriz de cambio de base entre estas tenga determinante $-1$.
Dado $J$ un $\O$-ideal fraccionario equivalente a $I_j$, fijamos
$r_J \in K^\times$ tal que $I_j = rJ$ y $N(r_J) > 0$ si $I_j$ y $J$
son estrechamente equivalentes. Vamos a asociarle a $J$ una $\Z$-base $B_J$;
si $N(r)>0$ tomamos $B_J = rB_j^+$, en caso contrario $B_J = rB_j^-$.

De esta manera, tenemos una aplicación bien definida
\begin{equation}\label{def:xi}
  \Xi \colon \CL_{NP}(\O) \to M_f/\SL_n \Z, \qquad [J] \mapsto [L_{J}]_{B_J}.
\end{equation}

\begin{prop} La función \eqref{def:xi} es inyectiva.
\end{prop}
\begin{proof}Basta ver que si existen $r$ de
norma negativa e $I_j$ que no sea estrechamente
equivalente a $r I_j$, entonces $\Xi([I_j])
\neq \Xi([rI_j])$. Por el contrarrecíproco,
si $\Xi$ emvía ambas clases a la misma clase de conjugación,
existe $U \in \SL_n \Z$ y un isomorfismo $\O$-lineal
$\phi \colon I_j \to r I_j$ que es bases $B_j^+$ y $r B_j^-$
se representa por $U$. Recordemos además que $\phi$
está dado por la multiplicación por cierto $x \in K^\times$.
Se tiene luego el siguiente diagrama
\begin{center}
\begin{tikzcd}
I_j \arrow{r}{\cdot x} & r I_j \arrow{r}{\cdot r^{-1}} & I_j\\
\Z^n \arrow{r}{U}\arrow{u}{B_j^+} &
\Z^n \arrow{u}{r B_j^-}\arrow{r}{C_{B_j^+,B_j^-}} & \Z^n\arrow{u}{B_j^+}\\
\end{tikzcd}
\end{center}
La composición de la fila superior define el morfismo de multiplicación por
$x/r$ de $I_j$ en sí mismo. Como se representa en base $B_j^+$, debe ser
$N(x/r) = \det \phi = \det U \det C_{B_j^+, B_j^-} = -1$ y por lo
tanto $N(x) = -N(r) > 0$. Esto muestra que $I_j$ y $rI_j$ son estrechamente
equivalentes, concluyendo la prueba.
\end{proof}

En general \eqref{def:xi} no es sobreyectiva. Por ejemplo,
si $K = \Q(i)$ y $\O = \Z[i]$, como $N_{K/\Q} \ge 0$
se tiene que $\CL_{NP}(\O) = \CL(\O) = 0$. Sin embargo,
al no haber enteros de norma $-1$, no todo par de matrices enteras
con polinomio característico es conjugada por una matriz de $\SL_n \Z$.
Explícitamente: si tomamos $A = \begin{pmatrix}
0 & 1\\-1 & 0\end{pmatrix}$ y $B = \begin{pmatrix}
0 & -1 \\ 1 & 0\end{pmatrix}$, una matriz inversible $D =
\begin{pmatrix} a & b \\ c & d\end{pmatrix}$
que satisfaga $AD = DB$ debe cumplir $b =c$, $d =-a$ y
entonces $\det(D) = -(a^2+b^2) \le 0$.

\begin{prop} Si existe $x \in K$ de norma negativa,
la función \eqref{def:xi} es biyectiva.
\end{prop}
\begin{proof} Por el Lema \ref{lem:sl=gl}
y las Proposiciones \ref{prop:uni-sl=gl} y
\ref{prop:equiv-clnp=cl}, separando en casos según
si $\O^\times = PN(\O)$, se obtiene que si $K\neq NP(K)$
entonces $\M_f/\SL_n \Z$ y  $\CL_{NP}(\O)$ siempre tienen
el mismo carindal. Esto juntocon la inyectividad de $\Xi$
prueban lo pedido.
\end{proof}


\begin{bibdiv}
\begin{biblist}

\bib{KCd}{article}{
   author={Conrad, K.},
   title={Ideal classes and matrix conjugation over $\Z$},
   eprint={https://kconrad.math.uconn.edu/blurbs/gradnumthy/matrixconj.pdf},
}

\bib{ft}{book}{
   author={Fr\"{o}hlich, A.},
   author={Taylor, M. J.},
   title={Algebraic number theory},
   series={Cambridge Studies in Advanced Mathematics},
   volume={27},
   publisher={Cambridge University Press, Cambridge},
   date={1993},
   pages={xiv+355},
   isbn={0-521-43834-9}
}

\bib{lm}{article}{
   author={Latimer, Claiborne G.},
   author={MacDuffee, C. C.},
   title={A correspondence between classes of ideals and classes of
   matrices},
   journal={Ann. of Math. (2)},
   volume={34},
   date={1933},
   number={2},
   pages={313--316},
   issn={0003-486X},
   doi={10.2307/1968204},
}

\end{biblist}
\end{bibdiv}

\end{document}
