\documentclass[11pt,a4paper]{amsart}
\usepackage[margin=1in]{geometry}

\usepackage{etoolbox}
\apptocmd{\sloppy}{\hbadness 10000\relax}{}{}
\newcommand{\edit}[1]{\marginpar{\footnotesize{#1}}}
\newcommand{\comment}[1]{}

%Text formatting and language configuration
\usepackage{microtype}
\usepackage[spanish]{babel}
\usepackage[utf8]{inputenc}
\usepackage[frenchstyle,largesmallcaps,light,oldstylemath,fulloldstylenums]{kpfonts}
\usepackage{enumitem}
\usepackage{amsmath,amsthm,amssymb,amsfonts}
\usepackage{tikz-cd}

%Semantic Macros
\newcommand{\C}{\mathbb{C}}
\newcommand{\R}{\mathbb{R}}
\newcommand{\Q}{\mathbb{Q}}
\newcommand{\Z}{\mathbb{Z}}
\newcommand{\N}{\mathbb{N}}
\newcommand{\A}{\mathbb{A}}
\newcommand{\F}{\mathbb{F}}
\renewcommand{\P}{\mathfrak{P}}
\newcommand{\nodiv}{{\not | \ }}
\renewcommand{\emph}[1]{\textbf{\textit{#1}}}

%Math Operators
\DeclareMathOperator{\coker}{coker}
\DeclareMathOperator{\colim}{colim}
\DeclareMathOperator{\disc}{disc}
\DeclareMathOperator{\rk}{rk}
\DeclareMathOperator{\End}{End}
\DeclareMathOperator{\Gal}{Gal}
\DeclareMathOperator{\Cl}{Cl}
\DeclareMathOperator{\car}{car}
\DeclareMathOperator{\adj}{adj}
\DeclareMathOperator{\mcd}{mcd}
\DeclareMathOperator{\Pic}{Pic}
\DeclareMathOperator{\ICM}{ICM}
\DeclareMathOperator{\M}{M}
\DeclareMathOperator{\GL}{GL}
\DeclareMathOperator{\SL}{SL}
\DeclareMathOperator{\CL}{Cl}

\usepackage{listings}
\usepackage{xcolor}

\definecolor{thmcol}{RGB}{135, 24, 107}
\definecolor{citecol}{RGB}{135, 24, 107}
\definecolor{linkcol}{RGB}{135, 24, 107}
\definecolor{urlcol}{RGB}{135, 24, 107}

%\numberwithin{equation}{section}

\theoremstyle{plain}
\newtheorem{ej}{Ejemplo}
\newtheorem*{que*}{Pregunta}
\newtheorem{thm}[equation]{Teorema}
\newtheorem{prop}[equation]{Proposición}
\newtheorem{lem}[equation]{Lema}
\newtheorem{obs}[equation]{Observación}

\renewcommand\qedsymbol{\Large\color{thmcol}${\divideontimes}$}

\usepackage[colorlinks=true, pagebackref=true]{hyperref}
\hypersetup{
    colorlinks,
    citecolor=citecol,
    linkcolor=linkcol,
    urlcolor=urlcol
}
\usepackage{amsrefs}
\usepackage{url}


\renewcommand{\SS}{\mathscr{S}}
\renewcommand{\O}{\mathcal{O}}
\newcommand{\fQ}{\mathfrak{Q}}
\newcommand{\fR}{\mathfrak{R}}

\usepackage{tcolorbox}

%\usepackage[notcite,notref]{showkeys}

\title[El problema de conjugación para matrices enteras]{El problema
de conjugación para matrices enteras}
\author{Guido Arnone}

\date{}

\begin{document}

\maketitle

\begin{tcolorbox}[colback=thmcol!30, sharp corners, colframe=white]
\begin{thm}[\cite{lm}*{Theorem}]\label{thm:main} Sea $\alpha$
un entero algebraico y $\mathcal O = \Z[\alpha]$. Existe una correspondencia
biyectiva entre ideales fraccionarios de $\mathcal O$ y clases de
conjugación de matrices enteras con polinomio característico $m(\alpha, \Q)$.
\end{thm}
\end{tcolorbox}

\section{Órdenes e ideales fraccionarios}

Sea $K$ una extensión finita de $\Q$ y $\O_K$ su anillo de enteros. Un
\emph{orden} de $K$ es un subanillo que como $\Z$-módulo tiene
rango $[K:\Q]$. Notar que $\mathsf{Frac}(\mathcal O) = K$, ya que el primer
cuerpo está contenido en el segundo y ambos tienen la misma
$\Q$-dimensión. Un \emph{$\mathcal O$-ideal fraccionario} es
un $\mathcal O$-módulo $I \subset K$; siempre existe $x \in \mathcal O$
y un ideal $J \trianglelefteq \mathcal O$ tal que $I = \frac{1}{x} J$.
\begin{ej} Si $\alpha$ es un entero algebraico y $K=\Q(\alpha)$,
entonces $\Z[\alpha]$ es un orden de $\O_K$. En particular
$\Z[\sqrt{-5}]$ es un orden de $\Q(\sqrt{-5})$ que está contenido
propiamente en su anillo de enteros $\Z[\frac{1+\sqrt{-5}}{2}]$.
\end{ej}
Dos $\mathcal O$-ideales fraccionarios $I$ y $J$ se dicen \emph{equivalentes}
si $I = xJ$ para algún $x \in K \setminus \{0\}$. Esta es una relación
de equivalencia; notamos $\ICM(\mathcal O)$ al conjunto de clases de
equivalencia de ideales fraccionarios. La multiplicación de ideales define
una estructura de monoide en este conjunto; notamos $\Pic(\O)$
al grupo de elementos inversibles de $\ICM(\O)$. En general,
si $\mathcal O \ne \mathcal O_K$, no todo
ideal fraccionario es inversible.

Precisaremos los siguientes lemas sobre ideales fraccionarios
más adelante.

\begin{lem} \label{lem:hom=units}
Sea $K$ una extensión finita de $\Q$ y
$\mathcal O$ un orden de $K$.
Dos $\O$-ideales fraccionarios $I$
y $J$ son equivalentes si y sólo si son isomorfos como $\O$-módulos.
Más aún, isomorfismo $\O$-lineal $I \to J$ está dado por la multiplicación
por un elemento de $K\setminus \{0\}$.
\end{lem}
\begin{proof} Si $I = xJ$ para cierto $x \in K \setminus \{0\}$,
el morfismo $j \in J \mapsto xj \in I$ resulta un isomorfismo $\O$-lineal.
Recíprocamente, supongamos que tenemos un isomorfismo $\O$-lineal
$\varphi \colon I \to J$. Por la implicación ya demostrada, podemos suponer
que $I, J \subset \O$, es decir que $I$ y $J$ son ideales de $\O$.
Ahora, dado $x \in I$ no nulo, para cada $i \in I$ es
\[
\varphi(x)i = \varphi(xi) = x\varphi(i).
\]
Esto implica que $\varphi$ coincide con el morfismo dado por la multiplicación
por $\varphi(x)/x$. En particular tomando imágenes es
$J = \frac{\varphi(x)}{x} I$.
\end{proof}

\begin{lem} \label{lem:frac-range}
Sea $K$ una extensión finita de $\Q$. Si $\O$ es
un orden de $K$, todo $\O$-ideal fraccionario no nulo
tiene rango $[K:\Q]$ como $\Z$-módulo.
\end{lem}
\begin{proof} Sea $I$ un $\O$-ideal fraccionario, que salvo
isomorfismo $\O$-lineal (en particular, $\Z$-lineal) podemos suponer
contenido en $\O$. Tensorizando por $\Q$ a la sucesión exacta
$0\to I \hookrightarrow \O \twoheadrightarrow \O/I \to 0$
vemos que $\rk I = \rk \O$ si y sólo si $\rk \O/I = 0$.
Para ver esto último probaremos que $\O/I$ es finito:
dado $x \in I$ no nulo tenemos un epimorfismo
$\O/x \O \to \O/I$; podemos asumir entonces que $I = (x)$.
Finalmente, el mismo argumento que en el caso $\O = \O_K$ prueba que
el cociente $\O/x \O$ tiene cardinal $N_{K/\Q}(x)$.
\end{proof}

\section{El teorema de Latimer-MacDuffee}

En esta sección probamos el Teorema \ref{thm:main}.
Fijemos $\alpha$ un entero algebraico con polinomio minimal $f$
de grado $n$ y notemos $\mathcal O = \Z[\alpha]$ y $K = \Q(\alpha)$.

Observemos que para todo $\O$-ideal fraccionario $I$
la multiplicación por $\alpha$ define un morfismo
$\Z$-lineal,
\[
m_I \colon I \to I, \qquad x \mapsto \alpha x.
\]
Por el Lema \ref{lem:frac-range}, todo tal ideal $I$ es $\Z$-libre
de rango $n$; en paticular, dada una $\Z$-base $B$ de $I$, podemos considerar
la matriz $[L_I]_B \in \M_n \Z$ de $L_I$ en base $B$.
Si cambiamos la base por otra, digamos $B'$, entonces $[L_I]_B$
y $[L_I]_{B'}$ son conjugadas con matriz de conjugación
la matriz de cambio de base $C_{B,B'}$.

Por otro lado, si $J$ es un $\O$-ideal fraccionario equivalente a $I$,
por el Lema \ref{lem:hom=units} esto equivale a tener un
isomorfismo $\O$-lineal $\phi \colon I \to J$
dado por la multiplicación
por cierto elemento $\beta \in K \setminus \{0\}$. Se sigue de aquí que
$\varphi m_I = m_J \varphi$, pues
\[
\varphi(m_I(x)) = \varphi(\alpha x) = \beta \alpha x
= \alpha \beta x = m_J(\varphi(x))
\]
para todo $x \in I$. En particular, dadas $\Z$-bases $B$ de $I$ y $B'$ de $J$,
las matrices $[L_I]_{B}$ y $[L_J]_{B'}$ serán conjugadas con matriz de
conjugación $[\varphi]_{B, B'}$. (También se puede observar que si $J = xI$
para cierto $x \in K \setminus \{0\}$, entonces $[L_J]_{xB} = [L_I]_B$.)

Si $I$ es un $\O$-ideal fraccionario,
vamos a notar $[L_I]$ a la clase de conjugación de las matrices $[L_I]_B$
donde $B$ es una $\Z$-base de $I$. El conjunto de matrices de $\M_n \Z$
de polinomio característico $f$ será denotado $\M_f$; recordemos que
$GL_n(\Z)$ actúa allí por conjugación.
La discusión anterior prueba la siguiente proposición.

\begin{prop} Se tiene una función bien definida
\begin{equation}\label{def:lambda}
\Lambda \colon \ICM(\O) \to \M_f/\GL_n(\Z), \qquad [I] \mapsto [L_I].
\end{equation}
\qed
\end{prop}

El Teorema \ref{thm:main} será una consecuencia de que la función $\Lambda$
es biyectiva, como veremos a continuación.

\begin{prop} La función \eqref{def:lambda} es sobreyectiva.
\end{prop}
\begin{proof} Sea $A \in M_f$ y veamos que existe un $\O$-ideal
fraccionario tal que $\Lambda([I]) = A$. Como $\O = \Z[X]/(f)$,
y $f(A) = 0$ por el teorema de Cayley-Hamilton,
la multiplicación por $A$ define una estructura de $\O$-módulo
en $N:= \Z^n$ donde la multiplicación por $\alpha$ se identifica con
la multiplicación por $A$. Más aún, esta estructura
es una restricción de la estructura
de $K$-módulo que $A$ define sobre $M := \Q^n$.

Observemos que
\[
n = \dim_{\Q} M = \dim_K M \cdot \dim_\Q K = \dim_K M \cdot n,
\]
así que $\dim_K M = 1$. En consecuencia, existe un isomorfismo
$K$-lineal $\varphi \colon M \to K$, que se restringe
entonces a un isomorfismo $\O$-lineal $\varphi| \colon N \to \varphi(N)$.
Por definición $I := \varphi(N)$ es un $\O$-ideal fraccionario y
la multiplicación por $\alpha$ en $I$ tiene matriz $A$ en base
$\{\varphi(e_1), \ldots, \varphi(e_n)\}$. En particular
$\Lambda([I]) = [A]$.
\end{proof}

\begin{prop} La función \eqref{def:lambda} es inyectiva.
\end{prop}
\begin{proof} Supogamos que $[L_I] = [L_J]$, de forma que existen
una matriz $U \in \GL_n \Z$ y $\Z$-bases $B$ de $I$ y $B'$ de $J$
tal que $U [L_I]_{B} = [L_J]_{B'} U$. La matriz $U$ define
un isomorfismo $\Z$-lineal $I \to J$ que, al conmutar con la multiplicación
por $\alpha$, es además $\O$-lineal. Por el Lema \ref{lem:hom=units},
se tiene entonces que $[I] = [J]$.
\end{proof}

\section{Ejemplos}

\subsection{De matrices a ideales} Si $\O_K$ es monogenerado
y $\CL(\O_K) = 1$, todo par de matrices con polinomio característico
$f$ son conjugadas. Hagamos un ejemplo no trivial.

Consideremos $d = -5$ y $K = \Q(\sqrt{-5})$, $\O = \O_K = \Z[\sqrt{-5}]$.
Como consecuencia de la cota de Minkowski,
el grupo de clases de $\O$ es isomorfo a $\Z_2$, generado por
$I = (2,1+\sqrt{5})$. Tomemos como $\Z$-base de $(1)$ a
$\{1,\sqrt{-5}\}$. De esta forma, la multiplicación de $\sqrt{-5}$
tiene en esta base matriz $A =
\small\begin{pmatrix}0 & -5\\ 1 & 0\end{pmatrix}$.
Para $I$ tomamos la $\Z$-base $\{2,1+\sqrt{5}\}$;
como $2\sqrt{-5} = -2 + 2(1+\sqrt{-5})$ y $\sqrt{-5}(1+\sqrt{-5}) = -5 +
\sqrt{-5} = -3 \cdot 2 + (1+\sqrt{-5})$, la multiplicación
por $\sqrt{-5}$ en esta base tiene matriz $B = \small\begin{pmatrix}
-1 & -3\\ 2 & 1 \end{pmatrix}$.

Por el Teorema \ref{thm:main}, toda matriz entera $C$
de $2 \times 2$ que satisfaga $C^2 = -5 I$ es conjugada a $A$ ó $B$,
y estas dos últimas no son conjugadas.

\subsection{De ideales a matrices}

\subsection{Una matriz que no es conjugada a su transpuesta}

\section{Conjugación por $\SL_n \Z$}

Refinamos ahora nuestra pregunta inicial. Si $UA = BU$
para ciertas matrices $A,B \in \M_n \Z$ y $U \in \GL_n \Z$,
entonces $\det U = \pm 1$.

¿Cuándo son dos matrices enteras
conjugadas por una matriz de determinante $1$?

\begin{ej}

\end{ej}

Cuando $n$ es impar, esta relación no es más estricta que la ya
considerada: si $\det U = -1$ entonces $\det -U = -1$ y $(-U)A = B(-U)$.

\subsection{El \emph{narrow-class group}}

Un elemento $x \in K$ se dice \emph{positivo}
si $N_{K/\Q}(x) > 0$ y \emph{totalmente positivo}
si $\sigma(x) > 0$
para todo embedding real $\sigma \colon K \to \R$.
Notamos $K^+ \subset K$
al conjunto de elementos totalmente positivos y
$\O^+ = \O \cap K^+$.
Se define el \emph{narrow-class group} de $\O$ como
el grupo $\Cl^+(\O)$ de $\O$-ideales fraccionarios inversibles
módulo la relación de
equivalencia $I \sim J \iff I = xJ$ para cierto $x \in K$ totalmente
positivo.

\begin{obs} Observemos que $N_{K/\Q}(x)$ se puede ver como el
producto de las imagenes de $x$ a través de cada
\textit{embedding} $\sigma \colon K \to \C$. Algunos de ellos
se pueden correstringir a $\R$. Si $\sigma$ es un embedding complejo,
también lo es $\overline \sigma$, y entonces
$\sigma(x)\overline\sigma(x) = |\sigma(x)|^2 \ge 0$. El signo
de la norma depende entonces únicamente de los embeddings reales; en
particular, un elemento totalmente positivo es positivo.
\end{obs}

\begin{prop} Se tiene una sucesión exacta corta
\[
1 \to \O^\times/\O^+ \to K^\times/K^+ \to
\CL^+(\O) \to \CL(\O) \to 1.
\]
\end{prop}
\begin{proof} Toda clase $[I]$ de ideal fraccionario
en $\CL(\O)$ es imagen de la clase de igual representante en $\CL^+(\O)$;
esto define un epimorfismo $\pi \colon \CL^+(\O)\to\CL(\O)$. El núcleo consiste
de las clases $[x \O]$ donde $x \in K^\times$. En particular se tiene
un morfismo $x\in K^\times/K^+ \mapsto [x \O] \in \CL^+(\O)$.
Su núcleo son las clases $[x] \in K^\times/K^+$ que satisfacen
$[x\O] = [\O]$, esto es, que existe $y \in K_+$ tal que $x \O = y \O$;
en particular $x/y \in \O$ y de forma simétrica $y/x \in \O$.
Por lo tanto $x = y \cdot z$ con $z \in \O^\times$
y la clase de $x$ en $K^\times/K^+$ pertenece a $\O^\times$. Finalmente
el núcleo del morfismo $\O^\times \to K^\times/K^+$ inducido por la inclusión
$\O^\times \subset K^\times$ es $\O^\times \cap K^+ = \O^+$.
\end{proof}

\begin{obs} Observemos que si $x \in K^\times$, entonces $x^2 \in K^+$.
En particular $K^\times/K^+$ y $\O^\times/O^+$ son $2$-grupos.
\end{obs}

\subsection{Ejemplos en el caso cuadrático} Sea $d$ un entero positivo libre de cuadrados y
$d \not \equiv 1 \pmod{4}$. Tomando $K = \Q(\sqrt{d})$, es $\O = \O_K =
\Z[\sqrt{d}]$.



\begin{bibdiv}
\begin{biblist}

\bib{lm}{article}{
   author={Latimer, Claiborne G.},
   author={MacDuffee, C. C.},
   title={A correspondence between classes of ideals and classes of
   matrices},
   journal={Ann. of Math. (2)},
   volume={34},
   date={1933},
   number={2},
   pages={313--316},
   issn={0003-486X},
   doi={10.2307/1968204},
}

\end{biblist}
\end{bibdiv}

\end{document}
